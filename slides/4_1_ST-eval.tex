%\documentclass[mathserif]{beamer}
\documentclass[handout]{beamer}
%\usetheme{Goettingen}
\usetheme{Warsaw}
%\usetheme{Singapore}
%\usetheme{Frankfurt}
%\usetheme{Copenhagen}
%\usetheme{Szeged}
%\usetheme{Montpellier}
%\usetheme{CambridgeUS}
%\usecolortheme{}
%\setbeamercovered{transparent}
\usepackage[english, activeacute]{babel}
\usepackage[utf8]{inputenc}
\usepackage{amsmath, amssymb}
\usepackage{dsfont}
\usepackage{graphics}
\usepackage{cases}
\usepackage{graphicx}
\usepackage{pgf}
\usepackage{epsfig}
\usepackage{amssymb}
\usepackage{multirow}	
\usepackage{amstext}
\usepackage[ruled,vlined,lined]{algorithm2e}
\usepackage{amsmath}
\usepackage{epic}
\usepackage{epsfig}
\usepackage{fontenc}
\usepackage{framed,color}
\usepackage{palatino, url, multicol}
\usepackage{listings}
%\algsetup{indent=2em}


\vspace{-0.5cm}
\title{Model Evaluation and Information Criteria}
\vspace{-0.5cm}
\author[Felipe Bravo Márquez]{\footnotesize
%\author{\footnotesize  
 \textcolor[rgb]{0.00,0.00,1.00}{Felipe José Bravo Márquez}} 
\date{ \today }




\begin{document}
\begin{frame}
\titlepage


\end{frame}


%%%%%%%%%%%%%%%%%%%%%%%%%%%


\begin{frame}{Model Evaluation and Information Criteria}
\scriptsize{
\begin{itemize}
\item In the context of scientific models, there are two fundamental kinds of statistical error \cite{mcelreath2020statistical}:

\begin{itemize}\scriptsize{
 \item \textbf{Overfitting}, which leads to poor prediction by learning too much from the data.
\item \textbf{Underfitting}, which leads to poor prediction by learning too little from the data.}
\end{itemize}

\item There are two common families of approaches to tackle these problems.


\begin{itemize}\scriptsize{
 \item \textbf{Regularization}: a mechanism to tell our models not to get too excited by the data.
\item \textbf{Information criterio}: a scoring device to estimate predictive accuracy of our models.}
\end{itemize}

\item In order to introduce information criteria, this class must also introduce information theory.

 
\end{itemize}



} 

\end{frame}




\begin{frame}{Conclusions}
\scriptsize{

\begin{itemize}
\item Blabla
\end{itemize}


} 
\end{frame}


%%%%%%%%%%%%%%%%%%%%%%%%%%%
\begin{frame}[allowframebreaks]\scriptsize
\frametitle{References}
\bibliography{bio}
\bibliographystyle{apalike}
%\bibliographystyle{flexbib}
\end{frame}  









%%%%%%%%%%%%%%%%%%%%%%%%%%%

\end{document}
