%\documentclass[mathserif]{beamer}
\documentclass[handout]{beamer}
%\usetheme{Goettingen}
\usetheme{Warsaw}
%\usetheme{Singapore}
%\usetheme{Frankfurt}
%\usetheme{Copenhagen}
%\usetheme{Szeged}
%\usetheme{Montpellier}
%\usetheme{CambridgeUS}
%\usecolortheme{}
%\setbeamercovered{transparent}
\usepackage[english, activeacute]{babel}
\usepackage[utf8]{inputenc}
\usepackage{amsmath, amssymb}
\usepackage{dsfont}
\usepackage{graphics}
\usepackage{cases}
\usepackage{graphicx}
\usepackage{pgf}
\usepackage{epsfig}
\usepackage{amssymb}
\usepackage{multirow}	
\usepackage{amstext}
\usepackage[ruled,vlined,lined]{algorithm2e}
\usepackage{amsmath}
\usepackage{epic}
\usepackage{epsfig}
\usepackage{fontenc}
\usepackage{framed,color}
\usepackage{palatino, url, multicol}
\usepackage{listings}
%\algsetup{indent=2em}


\vspace{-0.5cm}
\title{Generalized and Multilevel Linear Models}
\vspace{-0.5cm}
\author[Felipe Bravo Márquez]{\footnotesize
%\author{\footnotesize  
 \textcolor[rgb]{0.00,0.00,1.00}{Felipe José Bravo Márquez}} 
\date{ \today }




\begin{document}
\begin{frame}
\titlepage


\end{frame}


%%%%%%%%%%%%%%%%%%%%%%%%%%%


\begin{frame}{Generalized and Multilevel Linear Models}
\scriptsize{
\begin{itemize}
\item In this class we will learn two powerful extensions to the linear model, which we have discussed extensively throughout this course.

\item The first extensions if the \textbf{Generalized Linear Model} (GLM) which allows the use of distributions other than Gaussian in the outcome variable.

\item These extensions can be particularly useful when our outcome variable  is binary or bounded to positive values.

\item \textbf{Multilevel models} (also known as hierarchical or mixed effects models), on the other hand, are are useful when there are predictors at different level of variation.

\item For example, in studying scholastic achievement, we may have information at different levels:  individual students  (e.g., family background), class-level information (e.g., teacher), and school-level information (e.g., neighborhood) \cite{gelman2013bayesian}.


\item Multilevel models are extend linear regression to include categorical input variable representing these levels, while allowing intercepts and possibly slopes to vary by level \cite{gelman2006data}.



\end{itemize}



}

\end{frame}


\begin{frame}{Generalized Linear Models}
\scriptsize{
\begin{itemize}
\item ghghg  gfhfg\cite{mcelreath2020statistical}
 
\end{itemize}



} 

\end{frame}




\begin{frame}{Conclusions}
\scriptsize{

\begin{itemize}
\item Blabla
\end{itemize}


} 
\end{frame}


%%%%%%%%%%%%%%%%%%%%%%%%%%%
\begin{frame}[allowframebreaks]\scriptsize
\frametitle{References}
\bibliography{bio}
\bibliographystyle{apalike}
%\bibliographystyle{flexbib}
\end{frame}  









%%%%%%%%%%%%%%%%%%%%%%%%%%%

\end{document}
