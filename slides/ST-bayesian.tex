%\documentclass[mathserif]{beamer}
\documentclass[handout]{beamer}
%\usetheme{Goettingen}
\usetheme{Warsaw}
%\usetheme{Singapore}
%\usetheme{Frankfurt}
%\usetheme{Copenhagen}
%\usetheme{Szeged}
%\usetheme{Montpellier}
%\usetheme{CambridgeUS}
%\usecolortheme{}
%\setbeamercovered{transparent}
\usepackage[english, activeacute]{babel}
\usepackage[utf8]{inputenc}
\usepackage{amsmath, amssymb}
\usepackage{dsfont}
\usepackage{graphics}
\usepackage{cases}
\usepackage{graphicx}
\usepackage{pgf}
\usepackage{epsfig}
\usepackage{amssymb}
\usepackage{multirow}	
\usepackage{amstext}
\usepackage[ruled,vlined,lined]{algorithm2e}
\usepackage{amsmath}
\usepackage{epic}
\usepackage{epsfig}
\usepackage{fontenc}
\usepackage{framed,color}
\usepackage{palatino, url, multicol}
\usepackage{listings}
%\algsetup{indent=2em}
\newcommand{\factorial}{\ensuremath{\mbox{\sc Factorial}}}
\newcommand{\BIGOP}[1]{\mathop{\mathchoice%
{\raise-0.22em\hbox{\huge $#1$}}%
{\raise-0.05em\hbox{\L
\usepackage{fontenc}
\usepackage{framed,color}
\usepackage{palatino, url, multicol}
\usepackage{listings}
%\algsetup{indent=2em}
\newcommand{\factorial}{\ensuremath{\mbox{\sc Factorial}}}
\newcommand{\BIGOP}[1]{\mathop{\mathchoice%
{\raise-0.22em\hbox{\huge $#1$}}%
{\raise-0.05em\hbox{\Large $#1$}}{\hbox{\large $#1$}}{#1}}}
\newcommand{\bigtimes}{\BIGOP{\times}}
\vspace{-0.5cm}
\title{Introduction to Statistical Inference}
\vspace{-0.5cm}
\author[Felipe Bravo Márquez]{\footnotesize
%\author{\footnotesize  
 \textcolor[rgb]{0.00,0.00,1.00}{Felipe José Bravo Márquez}} 
\date{ \today }
arge $#1$}}{\hbox{\large $#1$}}{#1}}}
\newcommand{\bigtimes}{\BIGOP{\times}}
\vspace{-0.5cm}
\title{Introduction to Bayesian Inference}
\vspace{-0.5cm}
\author[Felipe Bravo Márquez]{\footnotesize
%\author{\footnotesize  
 \textcolor[rgb]{0.00,0.00,1.00}{Felipe José Bravo Márquez}} 
\date{ \today }


\begin{document}
\begin{frame}
\titlepage


\end{frame}


%%%%%%%%%%%%%%%%%%%%%%%%%%%


\begin{frame}{Bayesian Inference}
\scriptsize{
\begin{itemize}
 \item The statistical methods that we have discussed so far are known as frequentist (or classical) methods.
 \item There is another approach to inference called Bayesian inference.
 \item In modest terms, Bayesian data analysis is no more than counting the numbers of ways
the data could happen, according to our assumptions.
\end{itemize}

} 
\end{frame}




%%%%%%%%%%%%%%%%%%%%%%%%%%%
\begin{frame}[allowframebreaks]\scriptsize
\frametitle{Bilbiografía}
%\bibliography{bio}
%\bibliographystyle{apalike}
\begin{thebibliography}{8}

\bibitem{Assaad2008}
L. Wasserman \emph{All of Statistics: A Concise Course in Statistical Inference}, Springer Texts in Statistics, 2005.
\end{thebibliography}

%\bibliographystyle{flexbib}
\end{frame}









%%%%%%%%%%%%%%%%%%%%%%%%%%%

\end{document}
